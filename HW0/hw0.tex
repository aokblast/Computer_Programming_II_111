\documentclass{report}
\usepackage{listings}
\usepackage{hyperref}
\title{Computer Programming Homework 0}

\author{Sheng-Yi Hong}

\date{\today}

\begin{document}

\maketitle

\tableofcontents

\chapter{Requirement}

I recommend to use any Unix-Like OS (e.g. FreeBSD, MacOS, Linux) to finish this
homework. On windows, you can use either Virtual Machine or WSL to install Linux over Windows.

This homework requires \textbf{google test} package. Google test is a unit test
library on C/C++. It helps us build test system over this homework so that you can check
if you write code correctly. It is avaliable on most of the operating system.
If you use Ubuntu Linux, you can use command \textbf{sudo apt install
  libgtest-dev -y} to install \textbf{google test} package. As for those who're using other systems, you may need to find the commands yourselves then.

\chapter{Review of C}

\section{Problem 1: Linked List}

Linked list is widely used in operating system kernel due to the high
performance on insert and delete elements compare with other data structure like
array. Linked list can be implemented from pointer, which we have learned in
Computer Programming - I.

\subsection{Requirement}

In this problem, you're asked to implement Linked List of \textbf{int32\_t} in C by
using array. Because you haven't learned structure yet, I would like to see you using array to emulate the linking relationship. You have to modify all of the functions in \textbf{list.c} to
comply with the requirement. There are many types of Linked List. In this
problem, I use the definition in C++
\href{https://en.cppreference.com/w/cpp/container/list}{std::list}. Not all of
the function in \textbf{std::list} is required. You only have to finish ones in
\textbf{list.c}. Of course, you can add some auxiliary function to help you
implement linked list.

After meeting the requirements above, you can run \textbf{make test} to test all of
your codes. After passing all of the tests, you finish this problem.

\section{Problem 2}

\chapter{Preview of C++}

\section{Problem 3}

\section{Problem 4}

\end{document}